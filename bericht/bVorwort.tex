\chapter{Vorwort}

Die Idee, Firm- und Software für einen Bicycle Computer zu schreiben, haben wir über den Research Assistenten vom InES, Herr Dario Dündar. Die Themen, Entwickeln einer Android-Applikation, Firm- und Software für einen Cortex M3 schreiben sowie das Layouten eins Print sprachen uns sofort an. Manuel König erarbeitete bereits in seiner Projektarbeit eine Android-Applikation und hat grosses Interesse, sein Wissen zu vertiefen. Zudem werden beim Entwickeln einer einfachen Leiterplatte wichtige Erfahrungen gesammelt. Katrin Bächli hat als Ziel ihre C-Kenntnisse und Software-Skills zu stärken. Da ist die Vertiefung in die Firmware eines Cortex-Prozessor attraktiv. Wir wussten, dass wir auf eine funktionstüchtige Projektarbeit, die eine Harvesterschaltung entwickelte, zurückgreifen können.

Während der Umsetzung verlagerten sich die Themen zunehmend Richtung Hardware. Die Harvesterschaltung ist optimiert worden, damit das Ziel: ein Prototyp, der bei 10 km/h Daten an die Android-Applikation sendet, erreicht werden kann.

In der ganzen Entwicklung wurden wir sehr gut betreut. In gemeinsamen, wöchentlichen Sitzungen nahm sich Prof. Dr. Marcel Meli und Research Assistent Dario Dündar Zeit. Sie setzten sich intensiv mit unseren Fragen auseinander und gaben sehr gute Vorschläge. Durch das Fachwissen konnten viele Fragen und Probleme gelöst werden. Wir lernten somit sehr viel durch dieses Bachelorarbeit. 

Wir möchten Dario Dündar und Marcel Meli nochmals explizit herzlich danken. Die Unterstützung vor allem von Dario war unglaublich, denn Probleme tauchten einige auf. Eine weitere wichtige Person ist Erich Ruff vom InES. Schnell kamen wir beim Messen an die Genauigkeitsgrenze. Er baute für uns einen Radimitator auf. So konnten wir genaue und reproduzierbare Messergebnisse hervorbringen. Weiter danken wir Herr Blum von Delectric GmbH, welcher uns die Leiterplatten für die Arbeit kostenlos zur Verfügung gestellt hat. Zudem wurde die Leiterplatte sehr schnell geliefert, was unnötige Verzögerungen in der Arbeit verhinderte.

In dieser Projektarbeit konnten die Themengebiete wie Elektrotechnik für die Bewegungsinduktion, Elektronik für die Schaltungoptimierung und das Leiterplatten-Design, die hardwarenahe Programmierung durch das Aufsetzen der Firmware und das Interrupt-Konzept. Das Programmieren der Android-Applikation brachte Manuel König durch seine Projektarbeit selbst mit.

In der Bachelorarbeit konnte das ganze Wissen des Elektrotechnikstudium angewendet werden: Von der Hardware bis zu Software konnten Sachkenntnisse eingesetzt werden.


Winterthur, 10. Juni 2016


Manuel König
Katrin Bächli


\begin{figure}[ht]
   \includegraphics[width=1\textwidth]{imag_vorwort/delectric_logo_gross.gif}
   \caption{Delectric GmbH Logo}
   \label{delectric_logo} 
\end{figure}




