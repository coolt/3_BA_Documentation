\chapter{Diskussion}

Am Ende der Arbeit steht ein funktionstüchtiger Prototyp eines Bicycle Computers zur Verfügung. Die Leistungsgewinnung ist um xxx \todo{Faktor} optimiert, der Aufbau auf eine Leiterplatte miniaturisiert, das Energy Management auf die minimale Fahrgeschwindigkeit von 10 km/h optimiert sowie neu werden auch Sensordaten an die benutzerfreundliche Android-Applikation gesendet.\\

Der Prototyp ist so entwickelt, dass zukünftige Teams den Code schnell verstehen, die Leiterplatte leicht aufteilen und weiterentwickeln können und die modulare Android-Applikation ist nach Belieben ausbaubar.\\

Die Minimalanforderungen der definierten Aufgabenstellung wurden alle erreicht. Von den optionale Aufgaben sind nachfolgende Anforderungen eingebettet worden: Die Harvesterschaltung wurde optimiert, das Energy Management ist auf verschiedene Geschwindigkeiten angepasst, zusätzliche Sensoren (Temperatur und Feuchtigkeit) und das Auftrennen der Verarbeitungsschritte in ''Daten speichern'' und bei genügend Energie ''senden''.\\

Als mögliche Weiterentwicklungen stehen insbesondere Energieverbrauchsoptimierungen an erster Stelle. Das Ziel ist, dass bis zur Präsentation des Bicycle Computer an der Nacht der Technik, die Sensoren mit weniger Energie ausgelesen werden, sodass die Versorgungsspannung bei Geschwindigkeiten unter 40 km/h nicht abstellt. Interessant ist die Auswirkung des Connected Modes bei höherer Geschwindigkeit. Bei tieferen Geschwindigkeiten als 40 km/h ist der Verbindungsaufbau nicht realistisch. Doch bei höheren Geschwindigkeiten könnten über den Connected Mode sicher Daten versendet werden.\\

Die entwickelte Leiterplatte kann für Schülerinnen und Schüler zum Experimentieren mit BLE verwendet werden. Bewusst wurden viele Testpunkte und ein Stecker für das Abgreifen der Signale implementiert. Montagelöcher für eine Befestigung sind vorhanden und Steckplätze für die Kondensatoren gewähren einen flexiblen Einsatz der Kondensatoren. Das TI-SensorTag zusammen mit dem EM8500 eignen sich gut zum Kennenlernen des Energy Managements.\\
\newpage
Für die Vorführung des fertigen Produkts ist ein Gehäuse in Planung. Dieses wird an der Verstrebung zwischen Lenker und Sattel montiert. Die Distanz zum Rad kann flexibel eingestellt werden, sodass der Prototyp sich nicht auf ein Fahrradmodell limitiert. Zum Endprodukt zählt auch eine professionelle Montage der Doppelmagnete an den Speichen. Als Letztes wird der Hohlraum zwischen der Leiterplatte und dem TI-SensorTag durch zwei Verstrebungen verstärkt, damit das Gerät Bodenschläge aushält. Das Endprodukt ist somit ein voll anwendungsfähiger Bicycle Computer.\\

Auf das Produkt sind wir Stolz. Dies nicht zu Letzt, da die Umsetzung nicht ganz einfach war. Der EM8500-Chip ist ein neues Produkt und verhält sich teilweise nicht stabil. Konkret musste Manuel König acht EM8500-Chips auf die Leiterplatten anlöten, weil immer wieder einer ausstieg. Das heisst, entweder konnte die Kommunikation über SPI nicht mehr stattfinden, Slave-Address-Error, oder V\_SUP startete trotz grosser Energie am Eingang nicht. Da der Chip nicht zuverlässig funktionierte, war es in der Entwicklung schwierig zu unterscheiden, ob in der Harvesterschaltung ein Fehler auftrat oder der Chip einen Defekt aufwies. Zu oft suchten wir den Fehler in der Hardware und nicht im Chip. Die zweite Herausforderung war die Benutzung des TI-SensorTags ohne RTOS. Diese rudimentäre Programmierung machte Spass. Mit Hilfe des Know-Hows am InES konnten die Probleme längerfristig gelöst werden. Da es Pionierarbeit ist, ging die Entwicklung des Codes nicht so schnell wie gewünscht vorwärts. Sinnvoll ist z.B. bei mehr Energie per SPI das Status Register des EM8500 auszulesen. In diesen 8 Bits steht der Zustand des Energiezustands der Speicher und welche Schwellwerte überschritten sind.\\

Abschliessend möchten wir sagen, dass die Unterstützung bei der Bachelorarbeit sowohl von Prof. Dr. Meli wie auch von Research-Assistent Dario Dündar sehr gut und zeitnah war. Die Probleme, die auf uns zukamen, waren viel grösser als erwartet. Da wir es super im Team hatten, ergänzten wir uns auf eine gute Art und Weise. Die Freude blieb trotz teilweiser technischen Schwierigkeiten nie aus. Durch diese Arbeit haben wir sehr viel gelernt und freuen uns, zukünftig als Ingenieurin und als Ingenieur zu arbeiten.



