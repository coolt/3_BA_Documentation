\chapter{Verzeichnisse}




\renewcommand{\bibsection}{\section{\refname}}  % add a number to the bibl section
\makeatletter
%\renewcommand*\bib@heading{ \section{\refname}}
\makeatother


\bibliography{BibTex/references}



\section{Glossar und Abkürzungen}\label{glossar}

\textbf{Clock Domain}\\
\forceindent Ein Bereich der Hardware, der mit demselben Takt läuft.

\textbf{GPIO}\\
\forceindent General Purpose Input and Output bezeichnet die externe Schnittstelle zum Sensortag. Alle Zustände, die das EM8500 sendet, gelangen über die GPIO zum Sensortag. Für den Bicycle Computer ist der Reed Switch-Eingang der wichtigste, da aus diesem die Geschwindigkeit berechnet werden kann.

\textbf{InES}\\
\forceindent Ist der Institutsname eines Instituts der ZHAW. InES steht für Institut of Embedded Systems.

\textbf{KO}\\
\forceindent Steht für Kathodenstrahl-Oszilloskop und bezeichnet ein Messgerät, dass Spannung im Zeitverlauf aufzeichnet.

\textbf{MCU}\\
\forceindent MCU steht für Microkontroller Unit und bezeichnet eine Einheit aus einem Prozessor und den Peripheriebausteinen. Eine MCU ist ein lauffähiges System um einen MicroProzessor herum.


\textbf{Power Domain}\\
\forceindent Basiert auf der Fähigkeit eines Prozessor Speisungsgebiete zur Verfügung zu stellen. Der Prozessor teilt seine Funktionalitäten in Gebiete ein, die separat ein- und ausgeschalten werden können.

\textbf{MPP}\\
\forceindent Maximum Power Point (MPP) bezeichnet in einer Leistungskurve den höchsten Punkt, also das Leistungsmaxiumum.

\textbf{MPPT}\\
\forceindent Versucht ein System, einen Input stets auf das Leistungsmaximum zu regeln, spricht man von Maximum Power Point Tracking. Tracking steht für Einfangen.

\textbf{MPP-Ratio}\\
\forceindent Bezeichnet die Auswertung des MPP auf Spannungsachse. Liegt das Leistungmaximum beim Kurzschluss, so ist die MPP-Ratio bei 0\thinspace\%, liegt sie bei Leerlauf, dann liegt die MPP-Ratio bei 100\thinspace\%. Üblicherweise liegt die MPPT-Ratio dazwischen.

\textbf{MPPT-Ratio}\\
\forceindent Einstellungswert eines Registers im EM8500-Chip. Durch den Wert gibt man vor, auf welchen MPP das System sich einstellen (tracken) soll.

\textbf{RTC}\\
\forceindent Der Real Time Clock kann im Cortex M3 bei ausgeschaltener CPU weiterlaufen.  

\textbf{State Machine}\\
\forceindent Heisst korrekt Finite State Machine und bezeichnet eine Konzept, bei dem aufgrund einer Kombination von Eingangssignalen, sich das System in einem bestimmten Zustand befindet. In jedem Zustand sind nur gewisse Inputs zulässig, ansonsten verbleibt das System in diesem Zustand. Folgt ein korrekter Input, wechselt das System in den entsprechenden Zustand. 

\textbf{UML}\\
\forceindent Die Unified Modeling Language (UML) ist ein Quasistandard, wie Prozesse abgebildet werden können. Die Sprache definiert Formen, aufgrund deren man weiss, ob es sich um eine Initialiserung, eine Entscheidung oder um eine Verarbeitung, etc. handelt.


% make the list of figures title a section title
% i could not get it to work in the cls
\makeatletter
\renewcommand\listoffigures{%
    \section{\listfigurename}%
      \@mkboth{\listfigurename}%
              {\listfigurename}%
    \@starttoc{lof}%
}
\makeatother

% create a list of figures with the label used for figure in front of each entry
% http://tex.stackexchange.com/questions/155177/how-to-add-the-word-figure-to-the-list-of-figures
{%
\let\oldnumberline\numberline%
\renewcommand{\numberline}{\figurename~\oldnumberline}%
\listoffigures%
}

% make the list of tables title a section title
% i could not get it to work in the cls
\makeatletter
\renewcommand\listoftables{%
    \section{\listtablename}%
      \@mkboth{\listtablename}%
              {\listtablename}%
    \@starttoc{lot}%
}
\makeatother

{%
\let\oldnumberline\numberline%
\renewcommand{\numberline}{\tablename~\oldnumberline}%
\listoftables%
}
