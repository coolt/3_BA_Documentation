\chapter{Resultate}
\label{ch_resultat}

Im Kapitel \ref{ch_vorgehen} ist die Umsetzung ausführlich erklärt. In diesem Kapitel geht es um die relevanten Resultate. Beim Harvester geht es um den produzierten Print, die Leistungsfähigkeit der Schaltung, und die Qualität des Signals am Harvesterausgang. Beim Energy Management wird die Leistungsabgabe nach der Regelung dokumentiert, und die Wirkung der Schwellwerte auf die Kondensatoren STS und LTS und auf die Speisung V\_SUP. Beim Power Management wird der finale Stand an Schlafzyklen zusammengestellt und bei der BLE-Applikation das User Interface gezeigt.
 
\section{Harvesterschaltung}

\subsection{Der Print}

Die erstellte Leiterplatte ist in den Abbildungen \ref{print_vorne} und \ref{print_rueckseite} zusehen. Auf der Ansicht von oben ist der Top-Layer inklusive der Bestückung zu sehen. Die Bestückung des Top-Layers beinhaltet die Harvesterschaltung (unten rechts), den EM8500 (oben rechts) und den Stecker (mittig links), welcher als Verbindung zum TI-SensorTag dient. Ebenfalls sind die vielen Testpunkte zu sehen, jedes Netz wurde mit einem Testpunkt ausgestattet. Oben links sind die Anschlüsse für die Energiespeicher zu sehen, hier wurden die beiden Kondensatoren über Litzen angeschlossen, da die ElKos keinen Platz zwischen dem Prototypen-PCB und dem TI-SensorTag finden. 

Auf der Ansicht von unten ist der Bottom-Layer inklusive der Bestückung zu sehen. Die Unterseite der Leiterplatte beherbergt die Spule und den Reed-Switch, welche auf der oberen Seite der Leiterplatte keinen Platz mehr gefunden haben. Der Reed-Switch hätte vielleicht noch Platz gehabt auf der oberen Seite, jedoch muss der Reed-Switch in der Nähe der Spule platziert werden, da hier der Magnet durchläuft.

\begin{figure}[ht]
    \includegraphics[width=0.5\textwidth]{4Resultate/imag/print_rueckseite.png} 
    \caption{Print Ansicht von unten}
    \label{print_rueckseite}
\end{figure}


\begin{figure}[ht]
    \includegraphics[width=0.5\textwidth]{4Resultate/imag/print_vorne.png} 
    \caption{Print Ansicht von oben}
    \label{print_vorne}
\end{figure}

\newpage  
\subsection{Leistung am Harvesterausgang}

Wie erwartet (siehe theoretische Grundlagen \ref{mpp_theorie_diff}) ändert sich bei einer Hardware mit einer Spule das Leistungsmaximum. Die Abbildung \ref{mpp_resultat_harvester} zeigt die reale MPP-Kurve des Prototypen. Bei verschiedenen Geschwindigkeiten ist der MPP an einem anderen Punkt, problematisch ist, dass die Einstellung im EM8500 nicht so genau eingestellt werden kann, es sind nur grobe Schritte im Bereich von 50 - 70 \% verfügbar. 

\begin{figure}[ht]
    \includegraphics[width=0.5\textwidth]{4Resultate/imag/MPPHarvester.png} 
    \caption{Leistungskurve Harvesterausgang (normalisiert)}
    \label{mpp_resultat_harvester}
\end{figure}

Die Leistung, welche der Harvester zur Verfügung stellen kann liegt bereits bei einer Geschwindigkeit von 10 km/h bei 24.77 $\mu$W. Dies ist die Durchschnittsleistung bei optimaler Belastung, es handelt sich nicht um die Spitzenleistung. Es ist ersichtlich dass die Leistung mit der steigender Geschwindigkeit zunimmt.

\begin{tabbing}
    Geschwindigkeit \quad\= Leistung Harvester \\[0.8ex]
    10 km/h  \> 24.77  $\mu$W \\
    15 km/h  \> 61.07  $\mu$W \\
    20 km/h \> 116.69 $\mu$W \\
    40 km/h \> 472.61 $\mu$W \\
\end{tabbing}  

Es ist zu beachten, dass diese Werte die Leistung des MPP sind, somit können diese Leistungen nur erreicht werden, wenn die Einstellung des EM8500-Chips dementsprechend eingestellt werden.


\newpage  % todel
\subsection{Verhalten des Harvesterausgangs}

Die Abbildung \ref{resultat_Harvester_Spannung_47uF} zeigt das Verhalten des Harvesterausgangs mit einem 47 $\mu$F Elko bei der Belastung durch den EM8500-Chip. Die Spannung ist über einen längeren Zeitraum betrachtet sehr instabil, diese Problematik wurde bereits im Kapitel \ref{Ausmessen der Auswirkung des Ausgangskondensators} beschrieben. Die Abbildung \ref{resultat_Harvester_Spannung_100uF} zeigt das reelle Verhalten des Harvesterausgangs mit einem 100 $\mu$F Elko. Es ist zu sehen, dass die Spannung am Harvesterausgang über einen längeren Zeitraum relativ konstant bleibt. Genaue Messdaten zu dieser Problematik sind im Messprotokoll \todo{Messprotokoll} zu finden.

\begin{figure}[ht]
    \includegraphics[width=0.5\textwidth]{4Resultate/imag/SpannungVCC_47uF.png} 
    \caption{Spannung VCC beim Harvesterausgang bei 15 km/h}
    \label{resultat_Harvester_Spannung_47uF}
\end{figure}

\begin{figure}[ht]
    \includegraphics[width=0.5\textwidth]{4Resultate/imag/SpannungVCC_100uF.png} 
    \caption{Spannung VCC beim Harvesterausgang bei 15 km/h}
    \label{resultat_Harvester_Spannung_100uF}
\end{figure}

\newpage  % todel

\subsection{Energie am EM-Chipausgang}

Die Abbildung \ref{energie_resultat_harvester} zeigt die berechnete Energie, die am Ausgang des EM-Chip abgegeben wird in Bezug zur Geschwindigkeit. Die Berechnung, wie aus dem Puls über die Zeit, bis dass VSUP wieder eingeschalten wird, ist im  Messprotokoll yyy dokumentiert. Die gewonnene Energie, die dem TI-SensorTag zur Verfügung steht ist:

\subsubsection*{Tabelle Leistung EM-8500 Ausgang}
\label{res_em_aus}
\begin{tabbing}
    Geschwindigkeit \quad\= Leistung EM8500\_out \\[0.8ex]
    10 km/h  \> 5.44   $\mu$W\\
    15 km/h  \> 20.91  $\mu$W\\
    20 km/h> \> 41.39  $\mu$W\\
    40 km/h> \> 170.75 $\mu$W\\
\end{tabbing}  


\begin{figure}[ht]
    \includegraphics[width=0.5\textwidth]{4Resultate/imag/ResultatLeistungGeschwindigkeit.png} 
    \caption{Maximale Leistung vs. Geschwindigkeit}
    \label{energie_resultat_harvester}
\end{figure}

\subsection{Wirkungsgrad des Prototypen}

Aus den zwei Leistungsmessungen wird graphisch die Energiegewinnung im Vergleich dargestellt (Abbildung \ref{zsmEnergyGewinn}). Die Werte beziehen sich auf die gemittelte Leistung über 10 s. Es ist zu sehen, dass die Energiegewinnung bei der Harvesterschaltung mit der Geschwindigkeit deutlich schneller zunimmt, als die Energie am Ausgang des EM-Chips. 

\begin{figure}[ht]
    \includegraphics[width=1\textwidth]{4Resultate/imag/EnergyGewinnNachStelle.png} 
    \caption{Energiegewinn Zusammengefasst nach Stelle in der Schaltung}
    \label{zsmEnergyGewinn}
\end{figure}

Aus den zwei Leistungsmessungen wurde der Wirkungsgrad bei den gemessenen Geschwindigkeiten berechnet (siehe Tabelle unten). Der Wirkungsgrad ist bei tiefen Geschwindigkeiten sehr schlecht (24.87\thinspace\%). Dies ist eine der zwei Ursache, dass trotz genug geernteter Energie, das TI-SensorTag für das Senden der Sensorwerte bei tiefer Geschwindigkeiten nicht wie erwartet genug Energie hat. Mit mehr Geschwindigkeit, erhöht sich nicht nur die geerntete Leistung rapide, sondern auch der Wirkungsgrad. Der Wirkungsgrad des EM8500 innerhalb des entwickelten Prototypen liegt bei 40 km/h  bei 41.01\thinspace\%.  

\subsubsection*{Tabelle Leistung und Wirkungsgrad }
\begin{tabbing}
    Geschwindigkeit \quad\= Leistung Harvester \quad\= Leistung EM8500\_out \quad\= Wirkungsgrad\\[0.8ex]
    10 km/h  \> 21.87  $\mu$W \> 5.44   $\mu$W \> 24.87\thinspace\%  \\
    15 km/h  \> 57.19  $\mu$W \> 20.91  $\mu$W \> 36.56\thinspace\%  \\
    20 km/h \> 114.67 $\mu$W \> 41.39  $\mu$W \> 36.09\thinspace\%  \\
    40 km/h \> 416.29 $\mu$W \> 170.75 $\mu$W \> 41.01\thinspace\%  \\
\end{tabbing}   



\section{Energy Management}

Bei dem Vorgehen ist die Einteilung der Energiezustände in LOW, MIDDLE und HIGH ENERGY beschrieben (\ref{v_energiezustand}). Die folgenden Abbildungen im Unterkapitel \ref{tiefes_v} zeigen diese Verhalten. Zuerst wird das Verhalten bei LOW und MIDDLE ENERGY beschrieben. Da interessiert die Ladezeit, bis dass wieder ein Paket gesendet wird. Bei beiden Zuständen kann sich der LTS nicht laden, da das Senden der Pakete alle Energie verbraucht. Erst bei höherer Energie, kann das Ladeverhalten von LTS angeschaut werden. Diese Abbildungen sind im Unterkapitel \ref{res_entladen}. Im letzten Unterkapitel stehen die finalen Schwellwerte und Kondensatorenwerte, auf denen diese Messresultate basieren.

\subsection{Verhalten bei tiefen und mittleren Geschwindigkeiten}
\label{tiefes_v}


Das Primärziel, dass der Energy Harvesting Powered Bicycle Computer bei 10 km/h BLE-Pakete erhält ist gemäss Abbildung \ref{paket_100kmh} gelungen. Das violette Signal ist die Speisung V\_SUP, die nach dem erreichen des Schwellwertes von V\_STS freigeschaltet wird. LTS lädt sich bei niederer Geschwindigkeit nicht. Die Zeit für das Laden beträgt rund eine Minute. Die Software dieser Version beinhaltet, dass bei der ersten Speisung der Drucksensor geweckt wird  und bei der zweiten Speisung das Paket versendet wird. So erhält die Fahrerin oder der Fahrer jede zweite Minute ein BLE-Paket mit aktuellen Werten.

\begin{figure}[ht]
   \includegraphics[width=0.5\textwidth]{4Resultate/imag/pic_3.PNG}
    \caption{Senden von Paketen bei 9 - 10 km/h}
    \label{paket_100kmh}
\end{figure}

Bei 16 km/h verkürzt sich die Ladezeit auf 10 s (siehe Abbildung \ref{paket_16kmh}). Die Fahrerin bzw. der Fahrer erhält alle 20 s einen aktuellen Wert. Trügerisch in diesem Bild ist, dass LTS bereits geladen ist. Dies entsteht nicht durch 16 km/h, sondern entstand durch vorangehende Messungen mit höheren Geschwindigkeiten. Auf die Ladezeit des STS hat dies keine Auswirkung.


\begin{figure}[ht]
   \includegraphics[width=0.5\textwidth]{4Resultate/imag/pic2.PNG}
    \caption{Senden von Paketen bei 16 km/h}
    \label{paket_16kmh}
\end{figure}

Bei 40 km/h werden rund 15 BLE-Pakete mit gesendet, bis dass V\_STS (gelb) unter den Schwellwert fällt und V\_SUP (viollet) abstellt (Abbildung \ref{paket_40kmh}). Danach dauert es nicht ganz 2 Sekunden, bis dass wieder nonstop gesendet wird, für die nächsten 2.5 Sekunden. Bei 40 km/h werden alle Sensoren im Ringbufferprinzip ausgelsen: beim ersten Paket wird der Druck aktualisiert, danach die Temperatur, dann die Luftfeuchtigkeit. LTS lädt sich bei dieser Geschwindigkeit nicht, da V\_STS nur bis zum Schwellwert zur Speisung von V\_SUP gelangt und nie höher steigt.


\begin{figure}[ht]
   \includegraphics[width=0.5\textwidth]{4Resultate/imag/pic4.PNG}
    \caption{Senden von vielen Paketen bei 40 km/h}
    \label{paket_40kmh}
\end{figure}


\subsection{Laden und Entladen des LTS}
\label{res_entladen}

Das zweite Ziel, dass die vom Long Time Storage gespeicherte Energie für das Versenden der BLE-Pakete genutzt wird, wird in diesem Unterabschnitt erklärt. Die Bedingung nach Konzept des EM8500, dass VSTS den Schwellwert von v\_appl\_max\_\_lo von 3.7 V erreicht (siehe Schwellwerte graphisch dargestellt in \todo{Link}). In der Konfiguration wurde dieser Schwellwert so nah wie möglich an der Schwelle zur Speisung V\_SUP mit 3.6 V gewählt. Die Idee ist, sobald einmal V\_SUP konstant gespiesen ist, beginnt LTS unmittelbar mit dem Laden. Die Abbildung \ref{lts_ein} zeigt die finalen Einstellungen, zum  möglichst raschen Laden von LTS.

\begin{figure}[ht]
   \includegraphics[width=0.5\textwidth]{4Resultate/imag/LTS_Ladeschwelle.PNG}
    \caption{Einschalten von LTS nach Schwellwert von 3.7 V}
    \label{lts_ein}
\end{figure}

Dies ist erst möglich, wenn VSUP konstant gespiesen wird. Denn nur in diesem Fall kann VSTS über den Schwellwert zur Speisung der Applikation von 3.6 V steigen und die Schwelle für die Speisung des LTS bei 3.7 V aktivieren. Dies gelingt, ohne Sleep-optimierung, ab einer Geschwindigkeit von 50 km/h. Die Abbildung \ref{vsup_konstant} zeigt dies bei 60 km/h. LTS hat sich bereits auf die Höhe des STS von 3 V geladen. Ab einer Höhe von 3.6 V laden und entladen sich die beiden Kondensatoren parallel. Sie befinden sich im Connected Mode (siehe Datenblatt S. \todo{Seite}). Der Connected Mode ist in Abbildung \ref{connected_mode} dokumentiert.

\begin{figure}[ht]
   \includegraphics[width=0.5\textwidth]{4Resultate/imag/pic_5.PNG}
    \caption{VSUP wird konstant gespiesen. LTS lädt sich}
    \label{vsup_konstant}
\end{figure}

\begin{figure}[ht]
   \includegraphics[width=0.5\textwidth]{4Resultate/imag/STS_LTS_connect.PNG}
    \caption{Connedted Mode der zwei Speicher}
    \label{connected_mode}
\end{figure}

Die letzte Abbildung zum Verhalten des LTS zeigt, wie STS und LTS im Parallelbetrieb sich entladen, sich V\_SUP sich halten kann und beide sich wieder parallel laden. 
\begin{figure}[ht]
   \includegraphics[width=0.5\textwidth]{4Resultate/imag/pic1.PNG}
    \caption{Paralleles Entladen der Kondensatoren}
    \label{parallel_entladen}
\end{figure}

Diese Bilder belegen, dass das Laden und Entladen von LTS gut funktioniert. Das Problem liegt darin, dass bei der (noch nicht sleep-optimierten) TI-SensorTag Version V4, die Sensoren zu viel Energie verbrauchen und V\_SUP die Energie vernichtet. Dadurch kommt es nicht zum Laden des LTS. Erst bei 50 km/h ist das Verhalten exemplarisch demonstiertbar. Auszugehen ist, dass bei einer sleep-optimierten TI-SensorTag-Version das Laden und Entladen von LTS bereits bei tiefen Geschwindigkeiten funktioniert.




\todo{correct space in table}
% Values from V1
\subsection{Verwendete Werte}
\label{werte}

Die oben aufgeführen Messwerrte sind mit der Software-Version V4 für das TI-Sensortag und den nachfolgenden, genannten Werten umgesetzt.

STS = 100 $\mu$F
LTS = 3300 $\mu$F


\subsubsection*{Tabelle Finale Konfiguration Schwellwerte }
\begin{tabbing}
    Register .............\quad\= Spannungswert \\[0.8ex]
    v\_bat\_max\_hi       \> 3.8 V \\
    v\_bat\_max\_lo       \> 3.7 V \\
    v\_bat\_min\_hi\_dis  \> 3.6 V \\
    v\_bat\_min\_hi\_con  \> 2.2 V \\
    v\_bat\_min\_lo       \> 2.1 V \\
    v\_appl\_max\_hi      \> 3.8 V \\
    v\_appl\_max\_lo      \> 3.7 V \\   
\end{tabbing}  



\section{Power Management}

Bei der Entwicklung werden die Energiewerte für die gemessen. Diese werden in der untenstehenden Abbildung \ref{r_bild_e_zusammenfassung} zusammen mit der Energielieferung am EM8500-Chip zusammengestellt. Diese Abbildung fasst zusammen, bei welchen Geschwindigkeit nach welcher Zeit, welche Daten gesendet werden. 

\subsection{Energieaufwand pro Aufgabe}

Als Zusammenfassung des Energieverbrauchs dokumentiert die Abbildung \ref{energie_zsm} den Energieverbrauch für das Senden von Daten ab dem ungeladenen Zustand der Speicher. Die Kurve E\_Sensortag nach 10 s ist das Produkt der gemessenen Leistungswerte am EM8500-Ausgang (siehe Tabelle \ref{resultat_Zsm_Energy}) aufgerechnet als Energiewert nach 10 s und der Addition des Sockelbetrags, für das einmalige Aufladen von STS auf den Schwellwert von v\_bat\_min\_lo mit 2.1 V. Die Umrechnung zum Energiewert stehen in der Tabelle Unterhalb der Abbildung \ref{r_bild_e_zusammenfassung}, da die Werte aus der Abbildung nicht präzis abgelesen werden können und die Berechnung des Sockelbetraqs von xxxxx $\mu$J in der untenstehenden Gleichung.

Auf der y-Achse sind die Minimalwerte für das ab Start Senden eines BLE-Pakets mit und ohne Sensordaten eingetragen. Die Initialisierung und das Refreshen sind als violetter Sockelbetrag eingetragen. Auf diesen wird das Berechnen bzw. Auslesend der Daten und das Versenden, grauer Balken, dazuaddiert. Die Werte beziehen sich auf die Tabelle der Energiemessungen der Version V4 des TI-SensorTags (\ref{erst_EMessungen}). Die zweite Tabelle unterhalb der Abbildung \ref{r_bild_e_zusammenfassung} zeigt die Berechnung der Balken.



\begin{figure}[ht]
     \includegraphics[width=1\textwidth]{4Resultate/imag/EnergyVerbrauchZusammenfassung.png}\label{resultat_Zsm_Energy} 
     \caption{Energieverbrauch gem\"{a}ss Verarbeitungsaufwand für CPU}
     \label{r_bild_e_zusammenfassung}
\end{figure}


\subsubsection*{Energie Sensortag in 10 s}
\begin{tabbing}
    Geschwindigkeit \quad\= Energieabgabe EM8500\_out  \quad\= mit Sockelbeitrag STS\\[0.8ex]
    10 km/h      \> 54.4   $\mu$J        \> 274.9  $\mu$J\\
    15 km/h      \> 209.1  $\mu$J        \> 429.6  $\mu$J\\
    20 km/h>     \> 413.9  $\mu$J        \> 634.4  $\mu$J\\
    40 km/h>     \> 1707.5 $\mu$J        \> 1928.0 $\mu$J\\
\end{tabbing}  

\begin{equation}
  E_{Laden STS}= 100, \mu F \times \frac{1}{2}\, (2.1)^2 = 100, \mu F \times \frac{1}{2}\, 4.41 = 220.5, \mu F
\end{equation}


\subsubsection*{Tabelle Energieverbrauch nach Aufgaben}
\begin{tabbing}
    Funktionsblöcke ...................................................................................\quad\= Energieverbrauch \\[0.8ex]
    Einmaliges Laden der Kondensatoren auf TI-SensorTag        \> 63.2 $\mu$J\\
    Initialisierung + Refresh-Zyklen                           \> 77 $\mu$J + 13 $\mu$J = 90 $\mu$J\\
    Neustart = Laden Kondensatoren + Init + Refreshen          \> 63.2 $\mu$J + 90 $\mu$J = 153.2 $\mu$J\\
    Neustart + Geschwindigkeit und BLE-Paket                   \> 90 $\mu$J + 33 $\mu$J = 123 $\mu$J\\
    Neustart + Druck + Geschwindigkeit und Daten mit BLE-Paket \> 90 $\mu$J + 5 $\mu$J+ 43 $\mu$J = 138 $\mu$J\\
\end{tabbing} 



%Die Lösung ist das Aufsplitten von Aufgabenblöcken. Bei genug Energie, wird nicht ein Sensor gestartet, ausgelesen und seine Daten versendet, sondern der Sensor wird gestartet, dann geht das System schlafen, bis dass LTS voll ist. Dann folgt der nächste Aufgabe: die Werte auslesen. Es wird wieder geschlafen und beim dritten Mal genügend Energie wird das Paket versendet. Das Konzept des Schlaffens zwischen Aufgabeblöcken ist im theoretischen Teil im Unterkapitel \ref{pm_sleep} im Abschnitt \ref{schlafen_theorie} beschrieben . Die Abbildung \ref{arbeitsaufteilung} zeigt, das Aufteilen der Arbeitsschritte: Zuerst folgt die Initialisierung des Sensors, dann folgt das Auslesen der Daten und zum Schluss das Senden der Werte.
%
%\todo{Bild Arbeitsaufteilung }
%\begin{figure}[ht]
%    \includegraphics[width=0.4\textwidth]{idle.png}
%    %\includegraphics[width=0.1\textwidth]{4Resultate/imag/SpannungVCC.png} 
%    \caption{Aufteilen in drei Schritte bis zum Senden der Daten}
%    \label{arbeitsaufteilung}
%\end{figure}
%
%% ev. Grossaufnahme: BLE Energieverbrauch
%
%Das Ergebnis ist in der Abbildung \ref{Sensor_Energie} graphisch dargestellt. Es zeigt sich, dass das Senden von Sensordaten \todo{Wert eintragen Tabelle und Text} xx mal mehr Energie verbrauchen, als das Senden der Geschwindigkeit. Die schlechteste Bilanz hat das Auslesen des Status Registers aus dem EM8500-Chip. In diesem Byte ist hinterlegt, welche Schwellwerte des EM8500 überschritten wurden und dient dadurch auf einfache Art dem Abbilden des aktuellen Energiezustands (siehe Unterkapitel \ref{v_energiezustand}). 
%
%\begin{figure}[ht]
%    \includegraphics[width=0.4\textwidth]{idle.png}
%    %\includegraphics[width=0.5\textwidth]{4Resultate/imag/FakeLTSEntaden.png} 
%    \caption{LTS liefert Energie für die Arbeitspakete}
%    \label{Sensor_Energie}
%\end{figure}
%



Folgende Resultate werden mit dem produzierten Prototyp erzeugt:

\todo{ genauere Angaben nach Endmessungen}

\begin{enumerate}
    \item Das Übermitteln der Geschwindigkeit ist ab 10 km/h möglich
    \item Das Übermitteln aller Sensordaten (gestaffelt) ist ab xxx km/h möglich
    \item Das Übermitteln aller Sensordaten braucht bei einer Geschwindigkeit von 10 km/h xxxx s
    \item Das Übermitteln aller Sensordaten braucht bei 40 km/h xxx s
\end{enumerate}


Der Grund für die Schwierigkeit bei tiefer Geschwindigkeit liegt am hohen Sockelbetrag zum Starten des Systems. Der Short Time Storage muss zuerst auf die Höhe des Schwellwerts zur Speisung der Applikatin geladen werden. Dies bedeutet einen einmaligen Energieaufwand von 220 $\mu$J. Die Kondensatoren des TI-SensorTags müssen zu Beginn ebenfalls geladen werden. Dies sind weitere 63.2 $\mu$J. Zudem braucht die Initialisierung und das Refreshen zusammen in den ersten 10 s 90 $\mu$J. Zusammen ergibt dies einen Grundenergieverbrauch ab Start von 373.3 $\mu$J. Dies ist bei einer Geschwindikeit von 10 km/h erst nach xxx s möglich.
\todo{Zeit bei 10km/h eintragen. Abschlusssatz besser}





\newpage
\section{Ergebnisse BLE-Applikation}

Der Aufbau des TI-SensorTags zusammen mit dem selbst entwickelten Board wird \glqq Sensor\grqq\thinspace\ genannt. Dies, weil aus der Sicht einer Benutzerin oder eines Benutzers keine detaillierte Hardware besteht, sondern \glqq ein Sensor\grqq. Die Vereinfachung dient der Benutzerfreundlichkeit. Die Android-Applikation ist bewusst einfach aufgebaut, um die Benutzerin bzw. den Benutzer nicht zu verwirren. 

Der animierte Tachometer stellt die aktuelle Geschwindigkeit schnell und übersichtlich dar. Weitergehende Funktionen sind durch prägnante Namen selbsterklärend und der User sollte keine Mühe haben, die App ohne Lesen einer Anleitung zu verstehen.

\subsection{Applikationsstruktur}

Beim Öffnen der Applikation wird geprüft, ob Bluetooth aktiviert ist (siehe Abbildung \ref{permission}). Sollte Bluetooth nicht aktiviert sein, wird der User gefragt, ob Bluetooth aktiviert werden darf. Sollte der User die Aktivierung ablehnen schliesst sich die Applikation sofort.

\begin{figure}[ht]
    \includegraphics[width=0.5\textwidth]{4Resultate/imag/BLEBluetoothPermission.png} 
    \caption{Bluetooth Permission}
    \label{permission}
\end{figure}

Wird die Aktivierung der Bluetooth-Schnittstelle erlaubt, erscheint der Startbildschirm. Auf diesem befindet sich zentral der Tachometer (siehe Abbildung \ref{tacho}). Dieser zeigt Geschwindigkeiten von 0 – 90 km/h mit einer animierten Tachonadel an. Unterhalb des Tachometers werden die einzelnen Messwerte angezeigt und im untersten Teil des Startbildschirms befinden sich zwei Buttons, über die man Einstellungen vornehmen kann. Die Kontextmenus zu diesen Einstellungen werden in den nächsten zwei Abbildungen \ref{sensorauswahl} und \ref{einheiten} ersichtlich. 

\begin{figure}[ht]
    \includegraphics[width=0.5\textwidth]{4Resultate/imag/APPHomeScreen.png} 
    \caption{Startbildschirm der Applikation}
    \label{tacho}
\end{figure}

Wählt man auf dem Startbildschirm \glqq Sensor wählen\grqq, erscheint ein neuer Bildschirm. Auf diesem erscheinen nur die aktiven Bluetooth Geräte mit dem implementierten Prototypen-Filter. Der Bildschirm bildet die TI-SensorTag-Adresse ab. Der Verbund aus unserer neuen Leiterplatte und dem TI-SensorTag wird absichtlich als Sensor betittelt, da es sich für den User um ein Gerät handelt, welches Umgebungsdaten misst. Der User sieht das Gerät nicht in seinen Einzelteilen, sondern nimmt es als einen Sensor war. Es ist natürlich klar, dass nicht der Sensor (Temperatursensor, Drucksensor, etc) auf dem TI-SensorTag an sich ausgewählt wird, sondern die Adresse des Bluetooth-Chips. Jedoch ist das für einen User ohne technischen Hintergrund zu hochstehend. Jedes TI-SensorTag hat eine eigene Adresse und bei mehreren Prototypen im Raum, kann das entsprechende Gerät ausgewählt werden. Es muss noch überlegt werden, wie die Adresse benutzerfreundlicher gestaltet werden kann, da die Adresse nicht sofort mit einem spezifischen TI-SensorTag in Verbindung gebracht werden kann.

\begin{figure}[ht]
    \includegraphics[width=0.5\textwidth]{4Resultate/imag/BLEAdresseAuswaehlen.png} 
    \caption{TI-SensorTagauswahl}
    \label{sensorauswahl}
\end{figure}

Auf dem Startbildschirm befindet sich auch ein Konfigurationsknopf. In das Untermenu gelangt man, in dem man den Button "Einheiten + Einstellungen" auswählt.

\begin{figure}[ht]
    \includegraphics[width=0.5\textwidth]{4Resultate/imag/BLEEinheitenUndEinstellungenStart.png} 
    \caption{Einheiten und Einstellungen}
    \label{einheiten}
\end{figure}

Die Applikation stellt reiche Konfigurationsmöglichkeiten zur Verfügung (siehe Abbildung \ref{einheiten}. Zu jedem Sensor gehört ein Drop Down-Element, in dem die Einheiten ausgewählt werden können. Neben der Einheitenauswahl verfügt die App über die Möglichkeit, den Radumfang anzupassen und die Temperatur zu kalbrieren. 

Die Radgrösse, sprich der Radumfang, muss einstellbar sein, da die interne Geschwindigkeitsberechnung von diesem abhängt. Da die Radgrösse nicht normiert ist, konnte kein Drop Down-Element eingebaut werden. Die Zahl wird über die Tastatur in das Feld eingegeben, es können nur Zahlen eingegeben werden. Dies wurde implementiert um zu verhindern dass die Eingabe von der Applikation nicht interpretiert werden kann.

Bei der Temperaturanzeige ist eine Kalibrierung eingebaut. Dies, weil alle drei Temperatursensoren auf dem TI-SensorTag einheitlich zu hohe Werte ausgeben. Der zu hohe Temperaturwert liegt am Aufbau des Prototypen. TI-SensorTag und Print liegen nahe aufeinander und es entsteht Wärme im Zwischenraum. Der Benutzer oder die Benutzerin kann den korrekten Wert im Kalibrationsfeld eingeben. Die nächste Temperatur, die empfangen wird, wird mit dem eingetragenen Wert verglichen und ein Offset wird eingestellt. Wird keine Kalibration eingetragen, wird die Temperatur nicht kalibriert und der Offset bleibt unverändert.

\subsubsection*{Auswählbare Einheiten}
\begin{tabbing}
    Temperatur     \quad\= Geschwindigkeit            \quad\= Druck \\[0.8ex]
    Celsius (C)    \> Kilometer pro Stunde (km/h)\> Pascal (haP)\\
    Fahrenheit (F) \> Miles per hour (mph)       \> Bar (bar)\\
    Kelvin (K)     \>                            \> Atmosph\"{a}re (atm)\\
                   \>                      \> Pound-Force per sqare inch (psi)\\
                   \>                      \> Millimeter Quecksilber (mmHG)\\
\end{tabbing} 

  
Die vorgenommenen Einstellungen werden über den Button \glqq Speichern\grqq gesichert. Danach werden die eingelesenen Sensordaten in den ausgewählten Einheiten dargestellt.

\subsection{Paketverlust}

TI-SensorTag sendet im Advertising Mode drei Pakete pro Sendevorgang. Gemessen wird der Paketverlust bei einer Geschwindigkeit von 20 km/h (siehe untenstehende Tabelle und \todo{Messprotokoll erwähnen}). \todo{Tabelle referenzieren}  

\subsubsection*{Paketverlust BLE-Applikation}
\begin{tabbing}
    Messperiode \quad\= Gesendete Pakete \quad\= Empfangene Pakete \quad\= Paketverlust\\[0.8ex]
    1 min  \> 38   \> 35 \> 7.9\thinspace\%  \\
    1 min  \> 37   \> 31 \> 16.2\thinspace\%  \\
    2 min  \> 75   \> 64 \> 14.7\thinspace\%  \\
    2 min  \> 77   \> 65 \> 15.6\thinspace\%  \\
    5 min  \> 193   \> 168 \> 13.0\thinspace\%  \\
    5 min  \> 190   \> 158 \> 16.8\thinspace\%  \\
    10 min  \> 345   \> 301 \> 12.8\thinspace\%  \\
    10 min  \> 349   \> 302 \> 13.5\thinspace\%  \\
\end{tabbing} 

Da unsere Nutzung des Advertising Mode keine aktive Verbindung vorsieht, ist ein Paketverlust von rund 15\thinspace\% zu erwarten. Der Paketverlust ist normalerweise keine Problem, da der Empfänger bei feststellen eines Verlusts eine Aufforderung an den Sender schickt um das verlorene Paket neu zu senden. 

\subsection{Korrektheit der Daten}

Die Korrektheit der empfangenen Daten auf der Applikation wird mit einem visuellen Vergleich der Datenpakete im BLE-Sniffer von TI verglichen (siehe Abbildung \ref{sniffer}).  Die Daten im Sniffer entsprechen exakt den Daten, welche die Applikation empfängt. Der Inhalt der Daten wird somit unverfälscht übermittelt.


\todo{aktuelles Sniffer Bild} 

\begin{figure}[ht]
    \includegraphics[width=0.5\textwidth]{4Resultate/imag/sniffer.png} 
    \caption{Empfangende Daten des Sniffers}
    \label{sniffer}
\end{figure}

\begin{figure}[ht]
    \includegraphics[width=0.5\textwidth]{4Resultate/imag/sniffer.png} 
    \caption{Empfangene Daten der Applikation}
    \label{applikation_daten}
\end{figure}


\todo{Video zeiger (auf CD)}






