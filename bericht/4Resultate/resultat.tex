\chapter{Resultate}
\label{ch_resultat}

(In den Messprotokollen \todo{Auflisten der Messprotokolle} auf der CD sind diverse Energiemessungen dokumentiert.
 
\section{Harvesterschaltung}

\subsection{Leistung am Harvesterausgang}

Essentiell ist es zu wissen, wie viel Energie von der Harvesterschaltung zur Verfügung gestellt wird. Die Leistungskurve gibt Aufschluss wie viel Leistung bei verschiedenen Geschwindigkeiten gewonnen werden kann. Bei der eingetragenen Leistung handelt es sich um die maximal zur Verfügung stehende Leistung, also die Leistung im MPP. In Abbildung \ref{mpp_resultat_harvester} ist ersichtlich, dass die maximale Leistung mit erhöhen der Geschwindigkeit zunimmt.

\begin{figure}[ht]
    \includegraphics[width=0.5\textwidth]{4Resultate/imag/ResultatLeistungGeschwindigkeit.png} 
    \caption{Maximale Leistung vs. Geschwindigkeit}
    \label{mpp_resultat_harvester}
\end{figure}

\subsection{Verhalten des Harvesterausgangs}
Ein wichtiger Aspekt ist das reelle Verhalten des Harvesters bei Belastung mit dem EM-Chip. Der EM-Chip regelt den Eingang, bzw. den Ausgang des Harvesters, indem der Eingangswiderstand verändert wird. Somit soll der MPP erreicht werden, damit die Leistung immer maximal ist. Die Abbildung \ref{resultat_Harvester_Spannung} zeigt den Verlauf der Spannung am Eingang des EM-Chips über längere Zeit. Die Spannung wird auf ein stabiles Level geregelt und somit kann eine relativ konstante Leistung aufgenommen werden. Weitere Messungen und Werte können aus dem Messprotokoll \todo{Name heraussuchen} entnommen werden.

\begin{figure}[ht]
    \includegraphics[width=0.5\textwidth]{4Resultate/imag/MPPHarvester.png} 
    \caption{Leistungskurve (normalisiert) }
    \label{resultat_Harvester_Spannung}
\end{figure}


\subsection{Energie am EM-Chipausgang}

Die Energie, welche vom EM-Chip abgegeben wird, steigt mit der Geschwindigkeit. Hier wurde die Energie von einem Puls gemessen, d.h. vom Einschalten von VSUP bis zur Abschaltung von VSUP. Nach diesem Puls wird keine Energie mehr abgegeben, bis zum nächsten Puls

\begin{figure}[ht]
    \includegraphics[width=0.5\textwidth]{4Resultate/imag/SpannungVCC.png} 
    \caption{Spannung VCC bei 15 km/h}
    \label{resutat_emchip_spannung}
\end{figure}


\subsection{Wirkungsgrad des Prototypen}

Interessant ist die Betrachtung des Wirkungsgrades des EM-Chips. In der nachfolgenden Tabelle sind die Leistungen aufgezeichnet, welche mit den aktuellen Einstellungen des EM-Chips anliegen.

\subsubsection*{Tabelle Leistung und Wirkungsgrad } \todo{Tabellenwerte eintragen}
\begin{tabbing}
    Geschwindigkeit \quad\= Leistung Harvester \quad\= Leistung EM8500\_out \quad\= Wirkungsgrad\\[0.8ex]
    10 km/h  \> 21.87  $\mu$W \> 5.44   $\mu$W \> 24.87\thinspace\%  \\
    15 km/h  \> 57.19  $\mu$W \> 20.91  $\mu$W \> 36.56\thinspace\%  \\
    20 km/h> \> 114.67 $\mu$W \> 41.39  $\mu$W \> 36.09\thinspace\%  \\
    40 km/h> \> 416.29 $\mu$W \> 170.75 $\mu$W \> 41.01\thinspace\%  \\
\end{tabbing}   

Die Abbildung \ref{zsmEnergyGewinn} gibt einen Überblick, an welcher Stelle wie viel Energie vorhanden ist, bzw. zwischen welchen zwei Stellen wie viel Energie verloren ging.

\begin{figure}
    \includegraphics[width=1\textwidth]{4Resultate/imag/EnergyGewinnNachStelle.png} 
    \caption{Energiegewinn Zusammengefasst nach Stelle in der Schaltung}
    \label{zsmEnergyGewinn}
\end{figure}

Der Wirkungsgrad des EM8500 liegt bei 40 km/h  bei \todo{wirkungsgrad eintragen} und bei 10 km/h bei unserer Schaltung bei \todo{wirkungsgrad}.





\section{Energiemanagement}

%Beim Resultat spielt das Hard- und Softwaremanagment direkt ineinander, weshalb das Ergebnisse dieser zwei Aufgaben zusammen dargestellt werden.

Durch das korrekte Einstellen der Schwellwerte beim EM8500 (siehe Unterkapitel xxx) und die korrekten Ladewerte bei den Kondensatoren (siehe Unterkaptiel xxx), ist es möglich, dass sich der LTS-Kondensator bei einer Geschwindigkeit von YYY km/h lädt (siehe Abbildung xxx). Zudem entlädt sich LTS, sobald das Sensortag am Arbeiten ist (siehe Abbildung yyyy). Beim Energiemanagement ist es somit gelungen, die Schwellwerte und Kondensatorengrössen so einzustellen, dass die Funktionalitäten des EM8500-Chips voll ausgenutzt werden können.

\begin{figure}[ht]
    \includegraphics[width=0.1\textwidth]{4Resultate/imag/SpannungVCC.png} 
    \caption{STS und LTS laden sich}
\end{figure}

\begin{figure}[ht]
    \includegraphics[width=0.1\textwidth]{4Resultate/imag/SpannungVCC.png} 
    \caption{LTS liefert Energie für die Arbeitspakete}
\end{figure}


\section{Powermanagement}

Durch ein gutes Powermanagement (siehe Unterkapitel xxx) wurde es möglich, die energiestarken Aufgaben in Teilen zu erledigen. Die Abbildung xxxx zeigt, das Aufteilen der Arbeitsschritte: Zuerst folgt das Init, dann folgt das Auslesen eines Sensors, dann das Senden des Sensors. Die Aufgaben wurden aufgeteilt, weil alle drei Schritte in einem zu viel Energie verbraucht hätte, sodass VSUP zusammengebrochen wäre.

\begin{figure}[ht]
\includegraphics[width=0.1\textwidth]{4Resultate/imag/SpannungVCC.png} 
\caption{Drei Arbeitspakte bis zum Senden der Daten }
\end{figure}

%Ev.  Grossaufnahme: BLE Energieverbrauch

Die Abbildung \ref{resultat_E_Verbrauch_Verarbeitungsaufwand} zeigt den Energieverbrauch nach Verarbeitungsaufwand. Am wenigsten Energie benötigt das Berechnen der Geschwindigkeit über den RTC. Deutlich mehr Energie braucht das Auslesen der Sensor-Daten. Dies einerseits, weil die I2C-Kommunikation aufgebaut werden muss und weil die Sensoren eine gewisse Zeit brauchen, bis sie aktiv sind \todo{Aufwachzeit eines Sensors messen}. Die unterschiedlich verbrauchten Energiemengen entsprechen exakt den unterschiedlichen Startzeiten der Sensoren. 

\begin{figure}[ht]
\includegraphics[width=1\textwidth]{4Resultate/imag/EnergyVerbrauchNachKommunikation.png} \label{resultat_E_Verbrauch_Verarbeitungsaufwand} 
\caption{Energieverbrauch gemäss Verarbeitungsaufwand für CPU}
\end{figure}

Kombiniert man den Energieverbrauch mit der zur Verfügung stehenden Energie am Ausgang nach dem EM8500-Chips, können (siehe Abbildung \ref{resultat_Zsm_Energy}) folgende Schlussfolgerungen gezogen werden:

\begin{enumerate}
    \item bla
    \item bla
    \item bla
    \item bla
\end{enumerate}

\begin{figure}[ht]
\includegraphics[width=1\textwidth]{4Resultate/imag/EnergyVerbrauchZusammenfassung.png}\label{resultat_Zsm_Energy} 
\caption{Energieverbrauch gemäss Verarbeitungsaufwand für CPU}
\end{figure}


\section{Applikation}

Bilder der App

Sniffer Screenshot

Beweis, dass Pakete ankommen

ev. Video zeiger (auf CD)






