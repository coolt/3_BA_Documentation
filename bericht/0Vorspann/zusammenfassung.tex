\chapter*{Zusammenfassung}

Internet of Things, die Unterstützung des Menschen im Alltag durch intelligente Geräte, soll für einen Fahrradfahrer auf innovative Art nutzbar gemacht werden. Das  mobile Gerät soll durch Energy Harvesting gespiesen werden und bei einer Geschwindigkeit von 10 km/h Sensordaten mit Bluetooth Low Energy (BLE) senden. Die Arbeit baut auf einer Machbarkeitsstudie auf, die die gewonnene Energie mit dem Chip EM8500 verwaltet. Als Verarbeitungs- und Sendemodul wird das SensorTag von Texas Instruments der Serie Simple Link benutzt. Dieses Board beinhaltet einen Wireless MCU und den Low Power Cortex M3. 


Aufgabe der Arbeit sind das Entwickeln einer miniaturisierten Leiterplatte, die nicht grösser als das verwendete TI-SensorTag ist. Das Speichern der Energie und das Einstellen von Schwellwerten an den Speicherelementen, damit der Energiezustand im System bekannt ist. Die Freigabe der gesammelten Energie zur Nutzung aufgrund vordefinierten Schwellwerten. Der Code wurde aufgrund zweier Aspekte power-optimiert: Die Funktionen laufen ohne RTOS und beinhalten das Setzen von Registern.  Zudem schlafen grundsätzlich alle Power Domains. Sie werden nur für kurze Zeit für eine spezifische Aktion geweckt. Als Produkt steht neben der Hardware eine benutzerfreundliche Android Applikation zur Verfügung. Diese beinhaltet Einstellungen der Sensoren und einen ansprechenden Tachometer, der die Fahrgeschwindigkeit anzeigt. 


Anfangs wird der Aufbau der Machbarkeitsstudie in Betrieb genommen. Die bestehende Version sendet Geschwindigkeit ab 45 km/h und basiert auf einem fliegenden Aufbau.


Nach der Verbesserung der Harvesterschaltung, sodass bei 10 km/h rund 20 $\mu$W zur Verfügung stehen, wird der Print designt. Das Energy Managment im EM8500-Chip und die Firmware des TI-SensorTags werden komplett neu geschrieben. Die Schwellwerte beim Energy Management basieren auf dem ausgewerteten Leistungsmaximum des Harvesters und dem Ziel, konstant BLE-Daten bei 10 km/h zu senden. Bei der Firmware das TI-SensorTags werden über Sleep-Funktionen dem System genügend Zeit zum wieder Aufladen gegeben. 


Der Prototyp ist eine konfigurierbare BLE-Applikation, die bei 10 km/h jede 1.5 Minute Geschwindigkeits-, Druck-, Temperatur- und Luftfeuchtigkeitsdaten erhält. Bei 20 km/h werden die Daten nach 20 s und bei über 45 km/h konstant aktualisiert.

