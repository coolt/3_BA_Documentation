\chapter*{Vorwort}

Die Idee, Firm- und Software für einen Fahrrad-Computer zu schreiben, stammt vom Research-Assistenten vom InES, Herrn Dario Dündar. Die Themen vom Entwickeln einer Android-Applikation über das Schreiben einer Firm- und Software für einen Cortex M3 bis hin zum Layouten eines Prints sprachen uns, Manuel König und Katrin Bächli, sofort an. Ein Grund ist, dass das vorgeschlagene Projekt unsere unterschiedlichen Schwerpunkte bestens vereint, Manuel König wollte sich ins Schreiben einer Android-Applikation vertiefen und Katrin Bächli hat grosses Interesse an hardwarenaher Programmierung. 

Manuel König erarbeitete bereits in seiner Projektarbeit eine Android-Applikation und wollte das Wissen vertiefen. Auch interessierte ihn, von Grund auf eine einfache Leiterplatte zu designen. Katrin Bächli hatte generelles Interesse an Energy Harvesting und fand die Idee, ihre C-Kenntnisse durch dieses Projekt zu vertiefen verlockend. Wir wussten, dass wir auf eine funktionstüchtige Erstversion eines Fahrrad-Computers zurückgreifen konnten. Dieser wurde in einer Projektarbeit entwickelte und dient als Starthilfe. 

Während der Umsetzung des Prototypen verlagerte sich der Entwicklungsschwerpunkt zunehmend von der Software weg Richtung Hardware. Die Harvesterschaltung wird mehrfach optimiert, damit der Fahrrad-Computer schlussendlich bereits bei 10 km/h Daten an die Android-Applikation senden kann. 

Bei der Entwicklung des Fahrrad-Computers wurden wir sehr gut betreut. In gemeinsamen, wöchentlichen Sitzungen nahmen sich Prof. Dr. Marcel Meli und Research Assistent Dario Dündar Zeit. Sie setzten sich intensiv mit unseren Fragen auseinander und gaben sehr gute Vorschläge. Durch das Fachwissen konnten viele Fragen und Probleme gelöst werden. Wir lernten somit sehr viel durch diese Bachelorarbeit. 

Wir möchten Dario Dündar und Marcel Meli an dieser Stelle nochmals explizit danken. Prof. Dr. Marcel Meli kennt das Konzept des verwendeten EM8500 sehr gut und half uns, die richtigen Korrekturen vorzunehmen. Besonders in der Hardware-Entwicklung stellte er wichtige Fragen, die uns halfen, noch bessere Lösungen zu finden. Die Unterstützung durch Dario war unglaublich. Das TI-SensorTag konnte falsch reagieren wie es wollte, er gab nie auf und verhalf so, zu immer neuen Ansätzen und Lösungswegen. Ohne ihn würde der Fahrrad-Computer nicht bei 10 km/h fahren. Eine weitere wichtige Unterstützung erhielten wir durch Erich Ruff vom InES. Wir kamen beim Messen der Energie bei einer gewissen Geschwindigkeit an die Genauigkeitsgrenze. Er baute für uns einen Radimitator auf. So konnten wir genaue und reproduzierbare Messergebnisse erzielen. Weiter danken wir Herrn Blum von Delectric GmbH, welcher uns die Leiterplatten für die Arbeit kostenlos und zeitnah zur Verfügung gestellt hat. 

In dieser Projektarbeit konnte das ganze Wissen des Elektrotechnikstudiums angewendet werden: speziell vertieft worden sind die Themengebiete Elektrotechnik für die Bewegungsinduktion, Elektronik für die Schaltungoptimierung und das Leiterplatten-Design, die hardwarenahe Programmierung durch das Aufsetzen der Firmware und das Interrupt-Konzept. 

\begin{figure}[ht]
   \includegraphics[width=0.25\textwidth]{0Vorspann/imag/delectric_logo_gross.png}
   %\caption{Delectric GmbH Logo}
   \label{delectric_logo} 
\end{figure}

Weiter danken wir Herrn Blum von Delectric GmbH, welcher uns die Leiterplatten für die Arbeit kostenlos und zeitnah zur Verfügung gestellt hat.

In dieser Projektarbeit konnte das ganze Wissen des Elektrotechnikstudiums angewendet werden. Speziell vertieft worden sind die Themengebiete Elektrotechnik für die Bewegungsinduktion, Elektronik für die Schaltungoptimierung und das Leiterplatten-Design, die hardwarenahe Programmierung durch das Aufsetzen der Firmware und das Interrupt-Konzept. 


Winterthur, 10. Juni 2016


Katrin Bächli\\
Manuel König





